\documentclass{article}

\author{Manuel Hinz}

\title{Tagesdokumentation}

\begin{document}

\maketitle

\section{Vor B\"{o}nner}

\subsection{Probleme}

\begin{itemize}

\item Keine Pc Zugangsdaten

\item Pi und Sensoren nicht testbar, weil Surface kein ethernetport hat.

\end{itemize}

\subsection{Fragen/Antworten}

\begin{itemize}

\item D\"{u}rfen wir Pi an dem bildschirm und der Tastatur / Maus anschliessen ?  

\item Keine Weitere

\end{itemize}

\subsection{Tests}

Alle Motoren laufen Vorwärts und Rückwerts.

\subsection{Gemessene Werte}

\subsubsection{Motor 1}

\paragraph{Strom}

\begin{itemize}

\item $125 mA$ Leerlauf

\item $202 mA$ Max last

\end{itemize}

\paragraph{Spannung}

\subsubsection{Motor 2}

\paragraph{Strom}

\begin{itemize}

\item $149 mA$ Leerlauf

\item $203 mA$ Max last

\end{itemize}

\paragraph{Spannung}


\paragraph{Strom}

\begin{itemize}

\item $125 mA$ Leerlauf

\item $202 mA$ Max last

\end{itemize}

\paragraph{Spannung}


\subsubsection{Motor 3}
\subsubsection{Motor 3}

\paragraph{Strom}

\begin{itemize}

\item $125 mA$ Leerlauf

\item $202 mA$ Max last

\end{itemize}

\paragraph{Spannung}


\paragraph{Strom}

\begin{itemize}

\item $125 mA$ Leerlauf

\item $202 mA$ Max last

\end{itemize}

\paragraph{Spannung}




\paragraph{Strom}

\begin{itemize}

\item $123 mA$ Leerlauf

\item $202 mA$ Max last

\end{itemize}

\paragraph{Spannung}


\paragraph{Strom}

\begin{itemize}

\item $125 mA$ Leerlauf

\item $202 mA$ Max last

\end{itemize}

\paragraph{Spannung}



\subsection{Pi}

\begin{itemize}

\item Image auf die SD geflasht(f\"{u}r beide)

\item Grundlegender Funktionalit\"{a}tstest : LED auf dem Pi bliken lassen. 


\end{itemize}

\section{Nach Ankunft}

\subsection{Besprechung Dokumentation}

\begin{itemize}

\item Struktur besprochen

\item nicht unter 20 Seiten und andere Regeln (genauer Manuels Notizen vom 5.7.18)

\end{itemize}

\subsection{Fehlende Sachen PI}

\begin{itemize}

\item (micro) SD card min 8 gb --status : Leihgabe von Manuel Hinz

\item Energieversorgung (micro usb) --status : Leihgabe von Manuel Hinz

\end{itemize}

\subsection{Input PWM}

\begin{itemize}

\item PulsWeitenModulation

\item $U_{ein} = $Spannung wenn ein "AN" signal gesendet wird in $[V]$

\item $U_{aus} = $Spannung wenn ein "AUS" signal gesendet wird in $[V]$

\item $T = $Periodendauer in $[s]$

\item $T_{ein} = $ Zeit die $U_{ein}$ anliegt in $[s]$

\item $T_{aus} = $ Zeit die $U_{aus}$ anliegt in $[s]$

\item $U_m = U_{ein} * \frac{T_{ein}}{T}$ Durchschnitliche Spannung in $[V]$, 

angenommen : $U_{aus} = 0V$

\end{itemize}

\subsection{Aufgaben}

\subsubsection{Aufgabe 1} 

Gegeben
\begin{itemize}

\item $T = 10ms$

\item $U_{ein} = 5V$

\item $U_{aus} = 0V$

\item Efficiency $= 50/75/10 \%$ respectively $(a)/b)/c))$

\end{itemize}
 
Aufgabe : 

Zeitspannungsdiagramm zeichnen und $U_m$ ausrechnen

\subsubsection{Aufgabe 2}

Aufgabe: 

Diagramm gegeben $-> U_m$ herausfinden

\end{document}















