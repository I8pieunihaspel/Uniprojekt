\documentclass{article}

\title{Tagesdukumentation}

\author{Manuel Hinz}

\begin{document}

\maketitle

\section{Hilfe f\"{u}r die andere Gruppe}

\subsection{Pi}

\begin{itemize}

\item Sd nochmals geflasht + ssh datei erstellt.

\item Problem : Pi zeigt nur ein Buntes bild an 

\item Kabel gegeben 

\item Doku als Vorlage gegeben

\item https://github.com/raspberrypi/firmware/issues/946 gefunden sieht nach Einschicken aus

\end{itemize}

\section{F\"{u}r unsere Gruppe}

\subsection{VNC}

\begin{itemize}

\item VNC viewer installiert

\item Ip des Pi gefunden und connected. (IP : 192.168.43.28) 

\end{itemize}

\subsection{Motortreiber testen}

\begin{itemize}

\item Schaltung gefunden, aber Probleme, welche sich auf die Inputs des Pis beziehen bzw. auf die Simulation dieser. (Zusammenarbeit mit Aaron)

\item Scheint zu funktionieren

\item Ströme messen + Dokumentation mir weitergeben.

\end{itemize}

\subsection{Besprechungen}

\begin{itemize}

\item Manuel hat Salome grundlegendes LaTeX gezeigt.

\item Prozess des instalierens der Software vom GrovePi.

\item Problem : Suche nach, in der Doku erw\"{a}hnten, Potentiometers

\end{itemize}

\subsection{Messungen Mototreiber}

\subsubsection{HB1}
Werte:

Input1 : $7.47mV$

Input2 : $7.50mV$

Input3 : $7.47mV$

Input4 : $7.47mV$

\subsubsection{HB2}

Werte:

Input1 : $7.48mV$

Input2 : $7.49mV$

Input3 : $7.48mV$

Input4 : $7.47mV$

\subsection{GroovePi}

Software installiert

\subsubsection{Pythoncode}

\begin{itemize}

\item $https://github.com/DexterInd/GrovePi/blob/master/Software/Python/grove_infrared_distance_interrupt.py$ (Distance Beispiel)

\item $https://www.dexterindustries.com/GrovePi/programming/python-library-documentation/$

\item $https://github.com/DexterInd/GrovePi/tree/master/Software$ (Github repo)

\end{itemize}

\section{F\"{u}r n\"{a}chste Stunde}

ToDO

\begin{itemize}

\item Technikleiter festlegen

\item Technikleiter geht mit dem der anderen Gruppe hoch zum Raum mit Widerst\"{a}nden etc.vorstellen und zuh\"{o}ren

\item Sensoren ans laufen kriegen

\end{itemize}

\end{document}
