\documentclass{article}

\author{Manuel Hinz}

\title{Tagesdokumentat}

\begin{document}

\maketitle

\section{B\"{o}nner Input}

Tafelbilder sind im Verzeichnis 

\section{Sensoren Testen}

\begin{itemize}

\item Einen Abstandssensor getestet. Eingestellt auf 10 cm.

\item Andere werden dann nicht ganz getestet, sondern nur gleich eingestellt. 

\item Lichtsensor noch nicht.

\end{itemize}

\section{Programm planen}

\begin{itemize}

\item Programmiersprache Python vs Scala

\item Abstrakte Klassen :
\begin{itemize}

\item Sensor

\end{itemize}

\item Klassen : 

\begin{itemize}

\item Motor

\item MasterBrain

\item MotorSteuerung (H-Br\"{u}cke)

\item Helligkeitssensor (erbt von Sensor)

\item Abstandssensor (erbt von Sensor)

\end{itemize}

\item Interface / Trait : 
\begin{itemize}

\item Guardian

\end{itemize}

\end{itemize}

\section{Strecke}

\begin{itemize}

\item Strecke geplant

\item Werkstatt nachgefragt.

\item Ergebnis : Bauplan abgeben, danach wird die Strecke angef\"{a}hrtigt

\end{itemize}

\section{ToDo}


\begin{itemize}

\item Einleitung  und Problemstellung der Doku

\item Grundlegende Plannung der Strecke

\item Testen und kalibrieren aller Sensoren

\item Digitalisierung der Schaltpl\"{a}ne

\item \"{U}berarbeitung des Zeitplans

\item Schaltung zum Steurern des Motors

\item Plannung des Programms fortf\"{u}hren

\item Programmierung des Zugriffspunktes des Programmes zu Sensoren und Motoren.

\end{itemize}

\end{document}