\documentclass{article}

\author{Manuel Hinz}

\title{Tagesdokumentation}

\begin{document}

\maketitle


\section{Passw\"{o}rter}

\subsection{Pi}

Daten : 

Benutzer : pi

Passwort :  ge\"{a}ndert von raspberry zu iAtePi1

\subsection{Pc}

Daten : 

Benutzer : schule01

Passwort : schule18

\section{Organisation}


Klausuren : 

08.11.18

17.01.18

\section{Input}

\subsection{Spannungsteiler}

\subsubsection{Theorie}

Wichtige Formeln bei einfacher Reihenschaltung mit 2 Widerst\"{a}nden: 

\begin{itemize}

\item $U_G = U_1 + U_2$

\item $I = I_1 = I_2$

\item $\frac{U_a}{U_b} = \frac{R_a}{R_b}$

\item $R_G = R_1 + R_2$

\end{itemize}

2. Schaltung ist ein Spannugnsteiler

Tips :

\begin{itemize}


\item Die Paralellen Widerst\"{a}nde als einen im Ersatzschaltbild.

\item Man braucht keine Str\"{o}me um die Spannungen an den Widerst\"{a}nden auszurechnen.

\end{itemize}

\subsubsection{Aufgaben}

\begin{itemize}

\item eine einfache Aufgabe 2 Widerst\"{a}nde in Reihe, Str\"{o}me an den Widerst\"{a}nden ausrechnen.

\item einfacher Spannungsteiler Spannungen an den Widerst\"{a}nden $R_q$ und $R_l$ ausrechnen.

\item Variation von A1 nur mit zwei teilaufgaben

\item Aufgabe bezogen auf den einfachen Spannungsteiler : 

Ein Widerstand ist nicht gegeben, aber daf\"{u}r das Querstromverh\"{a}ltniss m, welches nicht gr\"{o}sser als 5 sein darf. Hier soll nun ein Widerstand nach der E12 reieh bestimmt werden, so das die 5 bzw. 0 V ann\"{a}hrend erreicht werden. 

\end{itemize}

\section{Sensoren}

\subsection{Abstand}

Hat geklappt, maximaler Abstand bei dem der Sensor zuverl\"{a}ssig ist : 21 cm gegen Manuels Formelsammlung

\subsection{Helligkeit}

Fehler :
	
	Code auf dem Pi
	
	Erwartet Array (subtype) bekommt Int  

\section{Landschaft}

Lego von B\"{o}nner f\"{a}llt weg, daher : 

\subsection{Andere Ideen}

\begin{itemize}

\item Pappe

\item Holz

\item Vorschlag von Felix (nicht sicher wie das heisst)

\item Bei der Werkstadt fragen (dann auf Vorschl\"{a}ge / m\"{o}glochkeiten reagiere)

\end{itemize}

\section{ToDo}

\subsection{Sensoren}

\begin{itemize}

\item Messen und Einstellen der Reichweite des IR distance Sensors.

\item Beispielcode des Lichtsensors zum laufen Kriegen.

\item Testen des Lichtsensors.

\end{itemize}

\subsection{Landschaft}

Landschaftsmaterial festlegen und Proben zum Testen der Sensoren beschaffen. (Absprache mit dem andren Team)

\subsection{Doku}

\begin{itemize}

\item Auf dem Laptop von Salome \LaTeX{} zum laufen bringen.

\item Dokumentationsvorbereitung f\"{u}r Messungen und Tests.

\end{itemize}

\end{document}
