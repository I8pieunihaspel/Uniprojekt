\documentclass{article}

\author{Manuel Hinz}

\title{Tagesdoku}

\begin{document}

\maketitle

\section{Recap und Plannung}

\subsection{Recap}

\begin{itemize}

\item Einweisung Robert

\item Alle Messungen so aufarbeiten, dass man es sp\"{a}ter in die Doku tuen kann.

\end{itemize}

\subsection{Plannung}

\begin{itemize}

\item Material besrogen zum testen des Abstandssensors

\item (interne)Messprotokolle anfertigen

\item Doku beginnen

\item Helligkeitssensor zum laufen bringen.

\end{itemize}

\section{Geschafft}

\subsection{Doku}

\begin{itemize}

\item Doku Struktur aufgebaut

\item Nach sprachlichen Spezifikationen gefragt : automatische (eng.) \"{U}berschriften sind ok. 

\end{itemize}

\subsection{Sensoren}

\subsubsection{Helligkeit}

Zum Laufen bekommen

\begin{itemize}

\item Problem bei analogRead, gel\"{o}st durch firmware update

\item Blick hinter die api : Code ist im schlechten bis sehr schlechten Zustand.

\end{itemize}

\subsubsection{Distanz}

Zum Laufen bekommen

\begin{itemize}

\item Lief zuvor.

\item Heute erst nach mehrmaligen Restarts zum Laufen gebracht.

\end{itemize}

\subsection{Praxis}

\subsubsection{Material zum Testen des Abstandssensors}

\begin{itemize}

\item Da Felix krank war : Wenig Plannung, Holz wurde geholt, und ist bereit zum Testen.

\end{itemize}

\subsubsection{Messprotokoll}

\begin{itemize}

\item Messprotokoll f\"{u}r die Messungen an den Motoren wurden erstellt

\item Diese dienen als internes Dokument um die Dokumentation zu unterst\"{u}tzen

\end{itemize}

\section{ToDo}

\begin{itemize}

\item Einleitung  und Problemstellung der Doku

\item Grundlegende Plannung der Strecke

\item Testen und kalibrieren aller Sensoren

\item Digitalisierung der Schaltpl\"{a}ne

\item \"{U}berarbeitung des Zeitplans

\item Bei Bedarf : neue Videos von Derek Banas

\item Schaltung zum Steurern des Motors

\item Plannung des Programms

\item Programmieren eines Testprogrammes zum testen der Sensoren

\item Programmierung des Zugriffspunktes des Programmes zu Sensoren und Motoren.

\end{itemize}

\end{document}