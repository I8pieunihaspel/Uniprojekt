\documentclass{article}

\author{Manuel Hinz}

\title{Grundwissen}

\begin{document}

\maketitle

\section{Raspbian}

Befehle f\"{u}r die Shell

\begin{itemize}

\item cd x: Change directory, den momentanen Ordner zum Unterordner x wechseln.

\item ls : list, zeigt alle Unterordner und Dateien im momentanen Ordner an.  

\item python3 x: das Script x, welches in python ist ausf\"{u}hren. 

\end{itemize}

\section{Python}

\begin{itemize}

\item class a(object): Anfang einer Klasse mit dem Namen a

\item class a(b): Anfang einer Klasse  mit dem Namen a, welche von der Klasse b erbt. 

\end{itemize}

\section{git}

\begin{itemize}

\item[•] in cmd eingeben $=>$ git clone Leerzeichen $"https://github.com/I8pieunihaspel/Uniprojekt"$ ein Leerzeichen $/home/pi/Desktop/repo_test$

\end{itemize}

\end{document}