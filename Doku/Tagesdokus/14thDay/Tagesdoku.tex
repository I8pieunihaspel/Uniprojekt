\documentclass{article}

\usepackage{hyperref}

\author{Manuel Hinz}

\title{Tagesdoukmentation}

\begin{document}

\maketitle

\section{Pi}

nach entlichen Versuchen von Unterschiedlichsten Sannungsquellen wurde das Problem des Pis der anderen Gruppe durch ein neueres Image gel\"{o}st.

\section{Simulation bzw. Programmierung}

\begin{itemize}

\item kurzes Besprechen der Funktionsweise und der Zusammenh\"{a}nge von den Vektoren "pos"(position), "vel"(velocity), "acc"(acceleration).

\item Implementation bremsen.

\item Erste implementation des R\"{u}ckwertseinparkens.

\item Implementation der darstellung der Reifen(fraglich ob die drinnen bleiben)

\item Collision detection Plannung.

\item Aufbau von einem Verst\"{a}ndnis des codes, welcher in  Javascript mit Hilfe von P5.js verfasst wurde (Murat und Robert)

\item Hilfestellung f\"{u}r die andere Gruppe.

\end{itemize}

\section{Doku}

\begin{itemize}

\item Hinzuf\"{u}gen von Versuchs- und Messungs-protokollen.

\item Im einzelnen : 
\begin{itemize}

\item H-Br\"{u}ckenmessungen.

\item Motormessungen.

\item Abstandssensorversuche.

\item Lichtsensorversuche.

\end{itemize}

\end{itemize}

\section{Wichtig f\"{u}r n\"{a}chste Stunde}

Angucken von Javascript und P5.js im Rahmen der Simulation.

\begin{itemize}

\item \url{https://developer.mozilla.org/bm/docs/Web/JavaScript}

\item \url{https://p5js.org/reference/}

\item \url{https://github.com/I8pieunihaspel/Uniprojekt/tree/master/Software/Simulation}

\end{itemize}

\section{ToDo}

\begin{itemize}

\item collision detection implementieren ?

\item Sensoren implementieren.

\item MasterBrain implementieren.

\item Doku und Pr\"{a}si voranbringen.

\item Ausbau der Grundwissensdatei.

\item Visualisieren unserer Strukturen auf Github.

\end{itemize}

\end{document}