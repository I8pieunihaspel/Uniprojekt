\documentclass{article}

\author{Manuel Hinz}

\title{Tagesdokumentation}

\begin{document}

\maketitle

\section{Github}

\subsection{Account}

\begin{itemize}

\item email : gitgood@web.de

\item password : erst IAtePie12 ge\"{a}ndert zu : iAtePi12

\end{itemize}

\section{Frage nach mehr PCs}

PC steht hier f\"{u}r PC / Laptop

Gebraucht : 

\begin{itemize}

\item 1 * PC f\"{u}r das Ansteuern des Pis.

\item 1 bzw. 2 * PC zum Programmieren.

\item 1 * PC zum Dokumentieren und f\"{u}r Notizen.

\item 1 * PC f\"{u}r Plannung / Recherche 

\end{itemize}

Vorhanden : 

\begin{itemize}

\item 1 * PC  : Uni

\item 1 * PC  : Manuel

\item 1 * PC Salome  

\end{itemize}

Das heisst es werden bis zu 4 gebraucht und es sind zwischen einem und dreien da.
\section{Doku}

\begin{itemize}

\item erste Version des Motivationsschreiben geschrieben.

\item erste Version der Zielsetzung geschrieben.

\item Protokoll zum Testen von  H-Br\"{u}cken geschrieben.

\item Protokoll zum Testen der Motoren geschrieben.

\item Anforderungen zu F\"{o}rmlichkeiten herrausgefunden.

\end{itemize}

\section{Pi}

Thema : GPIO pins zu einem digitalen output zu machen

\begin{itemize}

\item Nummerierung der Pins + Ansprechweise herrausfinden.

\item Testporgramm geschrieben

\item Kernerkenntnis der output muss immer wieder gesetzt werden.

\item Frage : Wie oft muss man den Output setzen ? 

\item Potentielle Frage : GND in ordnung ? hohe Spannung auch auf GPIO.LOW bzw. False


\end{itemize}

\section{Sensoren}

\subsection{Helligkeitssensoren}

\begin{itemize}

\item einen Sensor getestet.

\item Problem : Sensor muss sehr nah an die Oberfl\"{a}che um Hell von Dunkel unterscheiden zu k\"{o}nnen (zumindestens im Test : Tisch gegen Klebeband)

\item Bedenken : nochmals mit den wirklichen Oberfl\"{a}chen ausprobieren und gucken ob man das Problem reproduzieren kann.

\item L\"{o}sung : Ansteckplatz des Kabels uml\"{o}ten, so dass der Sensor n\"{a}her an den Boden kommt.

\end{itemize}

\section{Strecke}

Die Strecke wurde geplannt und bei der Werkstatt der Uni in auftrag gegeben. Ausserdem sind Details zur Beschaffenheit der Teststrecke bekannt.

\section{ToDo}

\begin{itemize}

\item Programme umbennenen und auf Github hochladen.

\item Orderstrukturen festlegen und umsetzen.

\item Plann des programmes genauer formulieren

\item Terminal Benutzung Dokumentieren

\item Kl\"{a}ren ob mehr Laptops mitgebracht werden sollen.

\item Programm zum checken und verbessern von \LaTeX Dokumenten schreiben.

\item Programm zum erstellen von Tagesdokus schreiben.

\item Quellen f\"{u}r erste Texte finden.

\item Bilder und Dokus auf github ???

\end{itemize}

\end{document}