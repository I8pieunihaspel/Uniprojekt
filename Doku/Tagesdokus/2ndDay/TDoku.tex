\documentclass{article}

\author{Manuel Hinz}

\title{Tagesdokumentation}

\begin{document}

\maketitle

\section{Intro}

\subsection{Praktikumszeit}

\begin{itemize}

	\item Mo 5-8 B\"{o}nner (Option bis zur 10. (1 Woche voher fragen))

	\item Di 5-8 Mei (alle zwei WOchen ?)

	\item Do 5-8 B\"{o}nner 

	\item Freitag vor den Herbstferien Pr\"{a}si

	\item Wirtschaft alle 2 Wochen (Freiarbeit zu Hause)

\end{itemize}

\subsection{Klausuren}

\subsubsection{1. Klausur Uniprojekt}

\begin{itemize}

	\item Gleichstromotor PWM

	\item Signalverarbeitung

	\item Sensor analog -> Digital

	\item Verst\"{a}rkung von Signalen (OP)
\end{itemize}

\subsubsection{2. Klausur}

\begin{itemize}

	\item Digital Technik

	\item Regelungstechnik

	\item Programmablauf

\end{itemize}

\section{Benotung}

$70$ Der Uniprojektnote

\begin{itemize}

	\item aktive Mitarbeit

	\item Fachgespr\"{a}che ab n\"{a}chste Woche immer DOnnerstags

	\item Lehrerinput Aufgabe m\"{u}ndliche Mitarbeit Montags

\end{itemize}


$30$ der Note:
Dokumentation (eine Note pro Gruppe)
\section{Achten auf}

\begin{itemize}

	\item Ruckeln bei der Beschleunigung

	\item Geschiwindigkeit

\end{itemize}

\section{Theorie}

Absprache mit B\"{o}nner am Montag \"{u}ber die Doku / Kapitel

\subsection{Gleichstromotoren}

\begin{itemize}

	\item Drehrichtung

	\item Spannungsquelle ? 

	\item Nenngr\"{o}ssen

	\item leerlauf

	\item max Last

	\item jeweils Strom und Spannung messen

	\item Datenblatt (H-Br\"{u}cke / Motor)

\end{itemize}

\section{Praxis / Aufgaben}

\subsection{Praxis}

Zu kaufen

\begin{itemize}

	\item Ethernet kabel ? 

	\item Sd / Micro SD

	\item Powerbank

\end{itemize}

\subsection{Aufgaben}

\begin{itemize}

	\item Grobe Planung der Hauptschaltung 

	\item Zusammenbau des Gestells

	\item Festlegung der Zeitplanung

\end{itemize}

\end{document}
