\documentclass{article}

\author{Manuel Hinz}

\title{Tagesdokumentation}

\begin{document}

\maketitle

\section{Organisation und Speicherung der Dateien}
Wir haben heute die Orderstruktur festgelegt in welcher wir alle Dateien speichern wollen. Diese Struktur wird auf Github zu finden sein.
\subsection{Struktur}

\begin{itemize}

\item Doku : 
\begin{itemize}

\item Abgabe : Ordner mit Unterordnern der verschiedenen Versionen

\item Quellen : Ordner mit allen Links zu genutzen Quellen 

\item Tagesdokus : Ordner mit einem unterordner zu jedem Tag

\end{itemize}

\item Hardware :
\begin{itemize}

\item Datenblätter : Ordner mit allen Datenblättern.

\item Messungen : Ordner mit allen Digitalisierten Messungen.

\end{itemize}

\item Software(in jedem Unterordner gibt es einen Ordner pro Sprache) :
\begin{itemize}

\item Endprogramm : Programm was am Ende auf dem Pi läuft + Klassen die das Hauptprogramm unterstützen.

\item Testen : Ordner in dem Scripte zum Testen von Funktionen liegen.

\end{itemize}

\item Zeitplan : Hier liegt immer der aktuelle Zeitplan + die Anwesenheitsliste für alle Tage

\end{itemize}

\subsection{Github}

\begin{itemize}

\item Repo erstellt. ($https://github.com/I8pieunihaspel/Uniprojekt$)

\item Struktur im Repo erstellt.

\item Alle Tagesdokus + aktuelle Version der Abgabedokumentation hochgeladen.

\item Code für Sensoren hochgeladen.

\end{itemize}

\section{Strecke}

\begin{itemize}

\item Strecke  + Extrateile bekommen

\item Doch keine Folie, daher ist die Farbe Braun.

\end{itemize}

\section{Sensoren}

\subsection{Helligkeitssensoren}

\begin{itemize}

\item Klappt mit geringem Abstand auch für den Unterschied Holz - Tape

\item Im Moment : beste Lösung ist das Umlöten des Anschlusses.

\end{itemize}

\subsection{Abstandssensoren}

\begin{itemize}

\item Funktionieren nur unter dem Tisch.

\item Nur einen Ausprobiert.

\item Lösung liegt vielleicht in der Karosserie, welche den Raum um den Sensor verdunkelt.

\item Alternativ : Umbau der Sensoren oder Verdunklung des Raumes.

\item Umbau der Sensoren detailiert im Abschnitt "Input". 

\end{itemize}

\section{Input}

\subsection{Funktion Infrarot Distanzsensor}

\begin{itemize}


\item Infrarot leuchte sendet Licht in eine Richtung.

\item Bauteil reagiert auf Helligkeit

\item Potentiometer als Spannungsteiler um Noise herrauszufiltern. 

\item Stärke des Signals nach Inverse Square Law : intesity $\propto \frac{1}{distance^2}$ 

\end{itemize}

\subsection{Die Fehlerquelle}

\begin{itemize}

\item Zuviel Noise.

\item Einmalige Einstellung des Noiselevels.

\end{itemize}

\subsection{Umbau}

\begin{itemize}

\item Infrarotleuchte ansteuerbar machen.

\item Potentiometer ausbauen und die Spannung direkt vom Pi geben lassen 

\item Funktion : Infrarotleuchte aus $\Rightarrow$ Messung vom Sensor $\Rightarrow$ Infrarotleuchte ab $\Rightarrow$ Messung vom Sensor $\Rightarrow$ Zweite Messung minus der ersten Messung ist ein Schätzwert der Intesität des Lichtes der Leuchte (proportional zum Abstand wie oben beschrieben).

\end{itemize}

\subsection{Probleme}

\begin{itemize}

\item nicht genug Anschlüsse für jede Infrarotleuchte und den Sensor (Umgebar durch Schaltung).

\item Zeitaufwendig 

\item Schwer

\end{itemize}

\section{Auto}

Auto zum fahren gebracht(ohne Ansteuerung vom Pi)

\section{Programmierung}

\begin{itemize}

\item (Abstrakte) Klasse Sensor geplannt und implementiert.

\item Klasse LightSensor geplannt und implementiert.

\item Klasse DistanceSensor geplannt und implentiert.

\item Genannten Code Kommentiert.

\item Aller Code ist noch in einer frühen Version.

\end{itemize}

\section{Quellen}

\begin{itemize}

\item Inverse Square Law : Why does this product equal $\pi$/2? A new proof of the Wallis formula for $\pi$. by 3Blue1Brown on Youtube (Stand 18.09.18)

\item Input von Herrn Meißner (Verständniss von Manuel Hinz)

\end{itemize}

\section{URGENT PROBLEMS}

\begin{itemize}

\item Internet wird immer von Manuel gestellt, geht nun nicht mehr, da Datenmengen immer größer werden, dadruch das quasi das Repo gecloned wird.

\item Potentielle Lösung : Kompletter Umbau des Repo, Andere Lösung für die Internet Verbindung des Pis. 

\end{itemize}

\section{ToDo}

\begin{itemize}

\item Plann des programmes genauer formulieren

\item Terminal Benutzung Dokumentieren

\item Kl\"{a}ren ob mehr Laptops mitgebracht werden sollen.

\item Programm zum checken und verbessern von \LaTeX Dokumenten schreiben.

\item Programm zum Erstellen von Tagesdokus schreiben.

\item Abstandssensor zum Laufen bringen.

\end{itemize}

\end{document}

















