\documentclass{article}

\author{Manuel Hinz}

\title{Tagesdokumentation}

\begin{document}

\maketitle

\section{Dokumentation}

\begin{itemize}

\item Erweiterung der Grundwissendatei

\item Neuer Abschnitt "Github" wurde geschrieben

\item Neuer Abschnitt "LaTeX" wurde geschrieben

\item neuer Abschnitt "Slack" wurde geschrieben

\item Beitr\"{a}ge zur HBr\"{u}cke, dem Ausblick und eine Bewertung des Praktikums wurden hochgeladen.

\end{itemize}

\section{Programmierung}

\begin{itemize}

\item Implementierung der Struktur der Motor und Motortreiber Klassen.

\item Hilfestellung bei dem anderen Team

\end{itemize}

\section{Gespr\"{a}ch}

\subsection{Daten}

\begin{itemize}

\item Vom Anfang (Bus um 3 nach) bis 12:10 Uhr

\item Mit allen Sch\"{u}lern

\end{itemize}

\subsection{Inhalt}

\begin{itemize}

\item Status beider Gruppen wurde vorgetragen

\item Flurgespr\"{a}che sind schlecht

\item Gruppen\"{u}bergreifende Gespr\"{a}che helfen der Moral, dem Fortschritt und der Qualit\"{a}t der Bewertung von einem Projekt.

\item Unsere Gruppe denkt ehr, dass wir das Projekt nicht fertig bekommen.

\end{itemize}

\section{Elektrotechnik}

\begin{itemize}

\item an den verschiedenen Grounds getestet. Hat nicht funktioniert.

\item nochmals die Ausgangsspannung gemessen welche immer noch 3.3 mV ist.

\item anderen Pi getestet, was wegen zu wenig Spannung nicht funktioniert hat.

\item Verstr\"{a}kung ?

\end{itemize}

\section{ToDo}

\begin{itemize}

\item Pi der anderen Gruppe zum laufen bringen und dann mit (und ggf. ohne) GrovePi testen.

\item Programmieren der Masterbrain Klasse und implementierung PWM.

\item Dokumentation und Pr\"{a}sentation

\item Neuer Zeitplan

\end{itemize}

\end{document}