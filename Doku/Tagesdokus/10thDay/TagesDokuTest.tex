\documentclass{article}

\author{Manuel Hinz}

\title{Tagesdokumentation}

\begin{document}

\maketitle

\section{Distance Sensor}

\subsection{Problem}

\begin{itemize}

\item Distanzsensoren funktionieren nur im Dunklen (unter dem Tisch)

\item Damit ist eine Funtion bei Tageslicht(wie bei der Pr\"{a}sentation) erschwert. 

\end{itemize}

\subsection{L\"{o}sungsversuche}

\begin{itemize}

\item R\"{o}hre um den Sensor bauen, um Noise zu verringern.

\item getestet :
\begin{itemize}

\item R\"{o}hre aus dem Karton des GrovePis

\item R\"{o}hre aus Papier(eng um den Sensor, nicht die Platine)

\item Kegelf\"{o}rmige Schutzh\"{u}lle, damit die H\"{u}lle nicht zum Grund der St\"{o}rung wird. 

\end{itemize} 

\item Karosserie bauen und testen ob sie genug Dunkelheit bietet.

\end{itemize}

\subsection{Weitere Ans\"{a}tze}

\begin{itemize}

\item Dunklere Umgebung 
\begin{itemize}

\item Flur. (geklappt)

\item Abgedunkelter Raum. (geklappt)

\end{itemize}

\end{itemize}

\section{Programmierung}

\begin{itemize}

\item Bereits programmierten Code(Sensor.py) in der Gruppe besprochen 

\item Pl\"{a}ne zum weiteren Vorgehen gemacht.

\item Github auf dem Surface aufgesetzt.

\end{itemize}

\section{Dokumentation}

Datei "Grundwissen" erstellt, in welchem Grundwissen informatischer Natur festgehalten wird, um alle dazu zu bef\"{a}higen den Pi zu bedienen und den Code zu verstehen.

\section{ToDo}

\begin{itemize}

\item Ding zusammenkleben.

\item Hauptdoku weiterschreiben.

\item Programm schreiben um die Tagesdoku vorzubereiten.

\item Programm zum vereinfach von \LaTeX schreiben.

\item Programmieren.

\item Auto durch Pi steurern.

\item PAP erstellen.

\item Github (wie benutzen, Regeln aufstellen, etc.)

\item Programmierung : Motor Klasse erstellen, zusammenarbeit mit Eteckniker.

\item Den Pi GitHub ready machen, git mit ssh key ?

\end{itemize}

\end{document}