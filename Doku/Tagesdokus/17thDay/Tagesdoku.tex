\documentclass{article}

\author{Manuel Hinz}

\title{Tagesdokumentation}

\begin{document}

\maketitle


\section{Geschaft}

\begin{itemize}

\item Beenden des Textes zur Teamdynamik ()

\item Beenden des Textes zu m\"{o}glichen Zuk\"{u}nftigen Entwicklungenen 

\item Beenden der Texte zu der Gesammten Schaltung und zum Blockschaltbild. (kinda)

\item Beenden des Textes zum Messen von Spannungen und Str\"{o}men.

\end{itemize}

\section{ToDo}

\begin{itemize}

\item Deckblatt mit Logos.

\item Blockausrutscher von Latex fixen.

\item mit Latex Helper bearbeiten.

\item Pr\"{a}sentation (keine Priorität)


\item Texte schreiben:
\begin{itemize}

\item Erkl\"{a}rung und Begr\"{u}ndung des Kaufes vom Pi

\item Erkl\"{a}rung und Begr\"{u}ndung des Kaufes vom GrovePi

\item Erkl\"{a}rung und Begr\"{u}ndung des Kaufes von den H-Br\"{u}cken

\item Erkl\"{a}rung des Motors

\item Erkl\"{a}rung und Begr\"{u}ndung des Kaufes von dem Abstandsensor (begonnen)

\item Erkl\"{a}rung und Begr\"{u}ndung des Kaufes von dem Helligkeitssensor (begonnen)

\item Erkl\"{a}rung PWM

\item Erkl\"{a}rung Spannungsteiler

\item Erkl\"{a}rung AD- / DA- Wandler

\end{itemize}


\end{itemize}

\end{document}