\documentclass{article}

\author{Manuel Hinz}

\title{Tagesdokumentation}

\begin{document}

\maketitle

\section{Was wurde gemacht / angefangen}

\begin{itemize}

\item Weitere Arbeiten an der Präsentation (siehe Unterkapitel Präsentation)

\item Text angefangen Teamdynamik

\item Text "Dateistrukturen" beendet

\item Text zu möglichen Zukünften des Projektes angefangen

\item Text zur Gesammtschaltung / Blockschaltbild angefangen

\item Text zu den Problem mit dem Pi angefangen.

\end{itemize}

\section{Präsentation}

\begin{itemize}

\item Umstieg von budget Powerpoint zu LibreOffice

\item Umwandelung der Datei und Beheben von Schwierigkeiten bei diesem.

\item Erlernen von Skills die für das Erstellen von PPPs benötigt werden(Murat)

\item Raussuchen von Design und Logo der Schule / Uni.

\end{itemize}

\section{ToDo}

\begin{itemize}

\item Texte schreiben:
\begin{itemize}

\item Erklärung und Begründung des Kaufes vom Pi

\item Erklärung und Begründung des Kaufes vom GrovePi

\item Erklärung und Begründung des Kaufes von den H-Brücken

\item Erklärung des Motors

\item Erklärung und Begründung des Kaufes von dem Abstandsensor

\item Erklärung und Begründung des Kaufes von dem Helligkeitssensor

\item Erklärung PWM

\item Erklärung Spannungsteiler

\item Erklärung AD- / DA- Wandler

\end{itemize}

\item Präsentation

\item Beenden des Textes zur Teamdynamik

\item Beenden des Textes zu möglichen Zukünftigen Entwicklungenen 

\item Beenden der Texte zu der Gesammten Schaltung und zum Blockschaltbild.

\item Beenden des Textes zum Messen von Spannungen und Strömen.

\end{itemize}

\end{document}