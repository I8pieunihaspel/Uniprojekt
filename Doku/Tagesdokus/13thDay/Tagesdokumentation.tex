\documentclass{article}

\author{Manuel Hinz}

\title{Tagesdokumentation}

\begin{document}

\maketitle

\section{Präsentation}

\begin{itemize}

\item Neue Zeit : 13 - 15 Uhr

\item Datum : 12.10.18

\end{itemize}

\section{Dokumentation}

\begin{itemize}

\item Text zum Tool LaTeXHelper geschrieben

\item Text zu Slack fertiggestellt 

\item Allgemeine Text Struktur von Bönner überprüfen lassen und Änderungen vorgenommen.

\item Struktur für die section "ET" vorläufig aufgestellt (siehe github)

\end{itemize}

\section{Weiterer Plan}

\subsection{Letzter Versuch}

Phillip bringt stuff mit und versucht alles zum Laufen zu bringen.

\subsection{Pyboard}

\begin{itemize}

\item Manuel hat ein pyboard

\item Könnte klappen, es wurde kurz über die Datenblätter geschaut.

\item allerdingds klappen dann unsere Sensoren nicht. D.h. Entweder blinde fahrt oder nur Beschleunigungssensoren.

\end{itemize}

\subsection{Simulation}

\begin{itemize}

\item Wahrscheinlichste Lösung.

\item Simulation mit Javascript und p5.js.

\item Grundlegendes Skript schon auf Github.

\item Notlösung.

\end{itemize}

\section{ToDo}

\begin{itemize}

\item Doku ET Texte.

\item Doku Drumherum.

\item Doku durch das Programm laufen lassen.

\item Simulation fertigstellen.

\item pyboard ausprobieren.

\item Präsentation weiterarbeiten.

\end{itemize}

\end{document}