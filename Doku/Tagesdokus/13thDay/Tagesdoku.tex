\documentclass{article}

\author{Manuel Hinz}

\title{Tagesdokumentation}

\begin{document}

\maketitle

\section{Pr\"{a}sentation}

\begin{itemize}

\item Neue Zeit : 13 - 15 Uhr

\item Datum : 12.10.18

\end{itemize}

\section{Dokumentation}

\begin{itemize}

\item Text zum Tool LaTeXHelper geschrieben

\item Text zu Slack fertiggestellt 

\item Allgemeine Text Struktur von B\"{o}nner \"{u}berpr\"{u}fen lassen und \"{A}nderungen vorgenommen.

\item Struktur f\"{u}r die section "ET" vorl\"{a}ufig aufgestellt (siehe github)

\end{itemize}

\section{Weiterer Plan}

\subsection{Letzter Versuch}

Phillip bringt stuff mit und versucht alles zum Laufen zu bringen.

\subsection{Pyboard}

\begin{itemize}

\item Manuel hat ein pyboard

\item K\"{o}nnte klappen, es wurde kurz \"{u}ber die Datenbl\"{a}tter geschaut.

\item allerdingds klappen dann unsere Sensoren nicht. D.h. Entweder blinde fahrt oder nur Beschleunigungssensoren.

\end{itemize}

\subsection{Simulation}

\begin{itemize}

\item Wahrscheinlichste L\"{o}sung.

\item Simulation mit Javascript und p5.js.

\item Grundlegendes Skript schon auf Github.

\item Notl\"{o}sung.

\end{itemize}

\section{ToDo}

\begin{itemize}

\item Doku ET Texte.

\item Doku Drumherum.

\item Doku durch das Programm laufen lassen.

\item Simulation fertigstellen.

\item pyboard ausprobieren.

\item Pr\"{a}sentation weiterarbeiten.

\end{itemize}

\end{document}