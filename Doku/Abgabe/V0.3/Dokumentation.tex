\documentclass{report}

\author{Manuel Hinz}

\title{Dokumentation des Unipraktikums}

\begin{document}

\maketitle

\tableofcontents
\newpage

\chapter{Einleitung}

\section{Motivation}
Im Moment sind selbstfahrende Autos so ziemlich das technische Thema. Universitäten rund um den Globus forschen an verschiedensten Technologien, welche das selbstfahrende Auto braucht. Ein Autohersteller nach dem anderen kauft entweder Startups zu dem Thema auf, oder investiert in eine eigene Forschungsabteilung. Als sich nun die Möglichkeit eines Praktikums im Rahmen des Elektrotechnikleistungskurses an der Bergischen Universität Wuppertal anbot, fiel die Wahl des Themas leicht. Das dieses Praktikum nicht nur eine gute Möglichkeit ist um den aktuellen Technischen Entwicklungen nachzueifern, zeigt sich schon an den verschiedenen Aufgaben die zu erledigen sind. Im Laufe des Projektes müssen nämlich Motoren mithilfe von H-Brücken und Pulsweitenmodulation gesteuert, ein Programm entworfen und ein Raspberrypi in Betrieb genommen werden. Neben dem technischen Wissen, was erlernt und angewendet werden muss (und sich praktischerweise relativ gut mit dem Abiturstoff deckt), wird hier auch ein Einblick in den Universitätsalltag erlangt.  
\section{Zielsetzung}
Es gibt viele mögliche Aufgaben, die man im Rahmen des Themas lösen könnte. Besonders interessant sind hier vor allem das finden von einem und das einparken auf einem Parkplatz, sowie das Teilnehmen am Straßenverkehr. Letzteres entfällt wegen der Schwierigkeit des Simulierens von anderem Verkehr. Deshalb ist die hier gewählte Aufgabe das finden und befahren einer Parklücke in einem straßenähnlichen Setting. Genauer gesagt wird das Auto eine Strecke entlangfahren und mit Hilfe von Sensoren eine Parklücke in die es hineinpasst identifizieren und dann dort einparken. Dies wird innerhalb von 16 90-minütigen Einheiten über einen Zeitraum von sieben Wochen. Ein weiteres Ziel ist es das für die Vollendung des Projektes nötige Wissen für alle Gruppenmitglieder verständlich zu bearbeiten und die Gruppe damit für das Abitur und weitere Projekte vorzubereiten.

\chapter{Theorie}

\section{Ausblick}

\section{Programmierung}

\section{Elektrotechnik}

\chapter{Praxis}

\section{Probleme und Problemlösungen}

\section{Menschliche Zusammenarbeit}

\chapter{Anhang}
\section{Quellen}
Test
\end{document}