\documentclass{article}

\author{Manuel Hinz}

\title{Besonderer Einsatz}

\begin{document}

\maketitle

In diesem Dokument sollen die au{\ss}erschulischen Eins\"{a}tze der einzelnen Sch\"{u}ler gew\"{u}rdigt werden, da in vieles in diesem Projekt entweder nach der Schulzeit endstanden ist, oder nur durch privates Equipment m\"{o}glich wurde.


\section{Aktivit\"{a}ten nach der Schulzeit}

\begin{itemize}

\item Murat : Pr\"{a}sentationsstruktur geplant.

\item Murat : Bekannt machen mit P5.js und Javascript f\"{u}r die Dokumentation.

\item Salome : Erlernen von den Grundlagen von \LaTeX .

\item Salome : Bereitmachen ihres Laptops durch das installieren  von Github und \LaTeX

\item Phillip : Alle Schaltpl\"{a}ne in dieser Dokumentation.

\item Phillip : Texte zum Grundlagenthema Messen.

\item Phillip : Aussuchen der zu bestellenden Teile

\item Robert : Text zum Helligkeitssensor.

\item Robert : Bekannt machen mit P5.js und Javascript f\"{u}r die Dokumentation.

\item Robert : Bereitmachen seines Laptops(Visual Studio Code, Github)


\item Manuel : Schreiben der 17 Tagesdokumentationen.

\item Manuel : Das Zusammenf\"{u}gen und Editieren der einzelnen Texte zu dieser Dokumentation.

\item Manuel : Aussuchen der zu bestellenden Teile

\item Manuel : Texte :  Gel\"{o}ste Probleme, Ausbessern von wir / uns in der Doku, Spannungsteiler  

\end{itemize}

\section{Privates Equipment}

\begin{itemize}

\item Salome : Mitbringen des eigenen Laptops.

\item Phillip : Mitbringen von (micro-)SDkarten, Netzger\"{a}ten, Abisolierzangen.

\item Robert : Mitnehmen der Materialien nach Ende des Projektes, Kleben von kaputgegangenen Teilen und das Mitbringen einer Powerbank f\"{u}r den Pi. Auch hat er seinen Laptop zur Uni mitgenommen.

\item Manuel : Mitbringen des eigenen Laptops, von einer (micro-)SDkarte und eines Netzteiles f\"{u}r den Pi, sowie das zur verf\"{u}gung stellen von Datenvolumen.

\end{itemize}

\end{document}