\documentclass{article}

\author{Manuel Hinz}

\title{Besonderer Einsatz}

\begin{document}

\maketitle

In diesem Dokument sollen die außerschulischen Einsätze der einzelnen Schüler gewürdigt werden, da in vieles in diesem Projekt entweder nach der Schulzeit endstanden ist, oder nur durch privates Equipment möglich wurde.


\section{Aktivitäten nach der Schulzeit}

\begin{itemize}

\item Murat : Präsentationsstruktur geplant.

\item Murat : Bekannt machen mit P5.js und Javascript für die Dokumentation.

\item Salome : Erlernen von den Grundlagen von \LaTeX .

\item Salome : Bereitmachen ihres Laptops durch das installieren  von Github und \LaTeX

\item Phillip : Alle Schaltpläne in dieser Dokumentation.

\item Phillip : Texte zum Grundlagenthema Messen.

\item Phillip : Aussuchen der zu bestellenden Teile

\item Robert : Text zum Helligkeitssensor.

\item Robert : Bekannt machen mit P5.js und Javascript für die Dokumentation.


\item Manuel : Schreiben der 17 Tagesdokumentationen.

\item Manuel : Das Zusammenfügen und Editieren der einzelnen Texte zu dieser Dokumentation.

\item Manuel : Aussuchen der zu bestellenden Teile

\item Manuel : Texte :  Gelöste Probleme,  

\end{itemize}

\section{Privates Equipment}

\begin{itemize}

\item Salome : Mitbringen des eigenen Laptops.

\item Phillip : Mitbringen von (micro-)SDkarten, Netzgeräten, Abisolierzangen.

\item Robert : Mitnehmen der Materialien nach Ende des Projektes, Kleben von kaputgegangenen Teilen und das mitbringen einer Powerbank für den Pi. 

\item Manuel : Mitbringen des eigenen Laptops, von einer (micro-)SDkarte und eines Netzteiles für den Pi, sowie das zur verfügung stellen von Datenvolumen.

\end{itemize}

\end{document}